\chapter{Exam 2 - Take Home Portion - Thursday, March 9\ts{th}, 2017}
\begin{prob}
A 1500 gallon tank initially contains 600 gallons of water with 5 lbs of salt dissolved in it. Water enters the tank at a rate of 9 gal/hr and the water entering the tank has a salt concentration of $\frac{1}{5} (1+cos(t))$ lbs/gal. Suppose the well mixed solution leaves the tank at a rate of 6 gal/hr.  
\renewcommand{\labelenumi}{\alph{enumi}}
\begin{enumerate}
    \item Draw a diagram of the tank problem.\\
    \item Set up the differential equation(s) needed to describe this problem, clearly describing all variables.
    \begin{step}
    Change in Volume:
    \begin{align*}
        V(0)&=100 \text{ gallons}\\
        \frac{dv}{dt}&=9\frac{\text{gal}}{\text{hr}}-6\frac{\text{gal}}{\text{hr}}=3\frac{\text{gal}}{\text{hr}}\\
        \frac{dv}{dt}&=3\\
        V(t)&=3t+V_0\\
        V(t)&=3t+600
    \end{align*}
    \end{step}
    \begin{step}
    Change in Salt:
    \begin{align*}
        \frac{dx}{dt}&=\left[ 9\frac{\text{gal}}{\text{hr}}*\frac{\frac{1}{5}(1+cost) \text{lbs}}{\text{gal}} \right] - \left[ 6\frac{\text{gal}}{\text{hr}} * \frac{x \text{lbs}}{\text{V(t) gal}} \right]\\
        \frac{dx}{dt}&=\left[ \frac{9}{5}(1+cost) \right] - \left[ \frac{6x}{3t+600}\right]
    \end{align*}
    \end{step}
    \item Solve the differential equation(s) you have developed in part 2.
    \begin{align*}
        \frac{dx}{dt}&=\left[ \frac{9}{5}(1+\cos t) \right] - \left[ \frac{6x}{3t+600}\right]\\
    \frac{dx}{dt} + \left[ \frac{6x}{3t+600}\right] &= \left[ \frac{9}{5}(1+\cos t) \right]\\
    \frac{dx}{dt} + \left[ \frac{2x}{t+200}\right] &= \left[ \frac{9}{5}(1+\cos t) \right]\\
    e^{\int \frac{2}{t+200}dt}&=e^{2\ln |t+200|}\\
    e^{\int \frac{2}{t+200}dt}&=(t+200)^2\\
    (t+200)^2*\frac{dx}{dt} + \frac{2x}{t+200} &=  \frac{9}{5}(1+\cos t) (t+200)^2\\
    \frac{d}{dt} \left[ (x)(t+200)^2 \right] &=  \frac{9}{5}(1+\cos t) (t+200)^2\\
    \int \frac{d}{dt} \left[ (x)(t+200)^2 \right] &= \int \frac{9}{5}(1+\cos t) (t+200)^2\\
    \frac{1}{3}(t+200)^3x&=\frac{9}{5}\int(1+\cos t) (t+200)^2\\
    \int(1+\cos t) (t+200)^2&=\int(t^2 + t^2 \cos(t) + 400 t + 400 \cos (t) + 40000 \cos (t) + 40000) dt\\
    \end{align*}
    \begin{align*}
     &= \int t^2 cos(t) \,dt + 400 \int t cos(t) \,dt + 40000 \int cos(t) \,dt + \int t^2 \,dt + 400 \int t \,dt + 40000 \int 1 \,dt\\
     &=t^2 \sin (t) - 2 \int t \sin(t) dt + 400 \int t \cos(t) dt + 40000 \int \cos(t) dt + \int t^2 dt + 400 \int t dt + 40000 \int 1 dt\\
      &= 2 t \cos(t) + t^2 \sin(t) + 39998 \int \cos(t) dt + 400 \int t \cos(t) dt + \int t^2 dt + 400 \int t dt + 40000 \int 1 dt\\
       &=2 t \cos(t) + 39998 \sin(t) + t^2 \sin(t) + 400 \int t \cos(t) dt + \int t^2 dt + 400 \int t dt + 40000 \int 1 dt\\
       &=2 t \cos(t) + 39998 \sin(t) + 400 t \sin(t) + t^2 \sin(t) - 400 \int \sin(t) dt + \int t^2 dt + 400 \int t dt + 40000 \int 1 dt\\
       &= 400 \cos(t) + 2 t \cos(t) + 39998 \sin(t) + 400 t \sin(t) + t^2 \sin(t) + \int t^2 dt + 400 \int t dt + 40000 \int 1 dt\\
        &= \frac{t^3}{3} + 400 \cos(t) + 2 t \cos(t) + 39998 \sin(t) + 400 t \sin(t) + t^2 \sin(t) + 400 \int t dt + 40000 \int 1 dt\\
        &= 200 t^2 + \frac{t^3}{3} + 400 \cos(t) + 2 t \cos(t) + 39998 \sin(t) + 400 t \sin(t) + t^2 \sin(t) + 40000 \int 1 dt\\
         &= \frac{t^3}{3} + 200 t^2 + t^2 \sin(t) + 40000 t + 400 t \sin(t) + 39998 \sin(t) + 2 t \cos(t) + 400 \cos(t) + c
    \end{align*}
    \begin{align*}
    \frac{1}{3}(t+200)^3x=\frac{9}{5} \left( \frac{t^3}{3} + 200 t^2 + t^2 \sin(t) + 40000 t + 400 t \sin(t) + 39998 \sin(t) + 2 t \cos(t) + 400 \cos(t) + c \right)\\
    \frac{x}{\frac{1}{3}(t+200)^3}=\frac{\frac{9}{5} \left( \frac{t^3}{3} + 200 t^2 + t^2 \sin(t) + 40000 t + 400 t \sin(t) + 39998 \sin(t) + 2 t \cos(t) + 400 \cos(t) + c \right)}{\frac{1}{3}(t+200)^3}\\
    x=\frac{\frac{9}{5} \left( \frac{t^3}{3} + 200 t^2 + t^2 \sin(t) + 40000 t + 400 t \sin(t) + 39998 \sin(t) + 2 t \cos(t) + 400 \cos(t) + c \right)}{\frac{1}{3}(t+200)^3}\\
    \end{align*}
    I definitely screwed the above math up. Next I would solve for $c$ by inputting the v(0)=100.
    \item Determine the time it will take for the tank to overflow.
    \begin{align*}
        \text{Overflow }V(t)&=1500\\
        1500&=3t+600\\
        900&=3t\\
        t&=300
    \end{align*}
    The tank will overflow at 300 hours.
    \item How much salt is in the tank when it overflows?
\end{enumerate}
\end{prob}
\begin{prob} Solve $y'= 5y +e^{-2x}y^{-2},y(0)=2$\\
Yeah, I have no idea.
\end{prob}
\begin{prob} Solve $y''+9y=8\sin t, y \left( \frac{\pi}{2} \right) =-1, y' \left( \frac{\pi}{2} \right) =1$
Same.
\end{prob}
