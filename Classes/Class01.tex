% !TEX root = /Users/us2009801/Documents/GitHub/DifferentialEquations/main.tex
\chapter{Class 1 - Thursday, January 19\ts{th}, 2017}
\section{\S 1 Differential Equations}

\begin{remark} $\dfrac{dx}{dt}=f'(x,t) \implies $rate of change of $x$ (state variable) with respect to time.
\begin{itemize}
\item Newton:$$y'=ky,y>0$$
\item Leibniz:$$\dfrac{dx}{dt}=kx,x>0$$

\begin{align*}
    k<0 & \implies \text{decay}\\
    k>0 & \implies \text{growth}\\
    \dfrac{dx}{dt} & =kx\\
    dx & =(kx)dt\\
    \frac{1}{x} dx & = k dt\\
    \textcolor{red}{\int} \frac{1}{x} dx & =\textcolor{red}{\int} k dt\\
    ln|x| & = kt+c\\
    \textcolor{red}{e}^{\textcolor{red}{(}ln|x|\textcolor{red}{)}} & = \textcolor{red}{e}^{\textcolor{red}{(}kt+c\textcolor{red}{)}}\\
    x & = e^{kt+c} = e^{c}e^{kt}\\
    x & = Ae^{kt} \leftarrow \text{A is a constant}
\end{align*}
\item verify:
\begin{align*}
    x & =Ae^{kt}\\
    \dfrac{dx}{dt} & =kAe^{kt} \leftarrow Ae^{kt} = x\\
    & = kx
\end{align*}
Also $x=0$ is a solution if there are no restrictions on x
$$\textcolor{red}{\dfrac{dx}{dt} = 0 = k*0 = kx}$$
\end{itemize}
\end{remark}
\section{\S 2 Definitions}
\begin{align*}
    \text{Set} & : \text{Collection of Objects}\\
    \text{Element} & : \text{Member of Set}\\\\
    \text{Domain of }f & : \text{Set of all independent value inputs of} f\\
    \text{Range of }f & : \text{Set of Dependent Values}\\
    f & : 0 \rightarrow \mathbb{R}, f(x)=y
\end{align*}
\begin{remark} Function of 2 independent variables:\\
If $(x,y)\in \epsilon$ maps to unique $z$, then $z=f(x,y)$ 
\end{remark}
\begin{remark} Region$\rightarrow \epsilon$: Set in plane if:\\
\begin{enumerate}
  \item Each point $p \in \epsilon$, $p$ is the center of a circle whose interior is also in $\epsilon$
  \item Every two points in $\epsilon$ can be connected by a curve entirely in $\epsilon$
\end{enumerate}
\end{remark}
\begin{imp:defn}{Implicit Function}{} Relation $f(x,y)=0$ defines $y$ as an implicit function of $x$ on $I:a<x<b$ if there exists $g(x)$ defined on $I$ such that $f[x,g(x)]=0$, $\forall x \in I$.
\end{imp:defn}
\begin{remark} $f(x,y)=16-x^{2}-y^{2}$ relation
Looking for $g(x)$ such that
\begin{align*}
    0 & = (16-x^{2})-[g(x)]^{2}\\
    g(x) & = \sqrt{16-x^{2}}\\
    0 & = 16-x^{2}-(16-x^{2})\\
    0 & = 16-x^2-16+x^{2}\\
    g(x) & = -\sqrt{16-x^{2}}
\end{align*}
\end{remark}
\begin{remark}Elementary Functions $\rightarrow$ Building Blocks to Work With
\begin{enumerate}
    \item Power Functions: $x^{n}$
    \item Root Functions: $x^{\frac{1}{n}}$
    \item Exponential Functions: $e^{x}$
    \item Log Functions: $log(x)$
    \item Trig Functions:
    \item Inverse Trig:
\end{enumerate}
\end{remark}
\section{\S 3 Differential Equations}
\begin{imp:defn}{Ordinary Differential Equation (ODE)}{} Let $f(x)$ define a function of $x$ on $I$:$a<x<b$ Then an ordinary differential equation (ODE) involves $x$, $f(x)$, and one or more derivatives of $f$
\end{imp:defn}
\begin{note} $I$ is our typical interval, & lazy, so won't write $a<x<b$.
\end{note}
\begin{remark}Order of Ordinary Differential Equation is order of the highest derivative
\begin{align*}
    f^{(4)}(x)+2f^{(2)}(x)+x & =0\\
    f''''(x)+2f''(x)+x & = 0
\end{align*}
\end{remark}
\begin{remark}Solution $x$ satisfies an ODE if when you substitute $x$ into ODE, it is a true statement.
\end{remark}
\begin{ex}
\begin{align*}
    y' & =e^{x}\\
    y & =e^{x}
\end{align*}
\end{ex}
\begin{ex}Differential Equation:
\begin{align*}
    (1+x^{2})y' & =xy\\
    y & = \sqrt{1+x^{2}}
\end{align*}
\end{ex}