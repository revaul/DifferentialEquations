% !TEX root = /Users/us2009801/Documents/GitHub/DifferentialEquations/main.tex
\chapter{Class 10 - Monday, February 27\ts{th}, 2017}
\section*{\S 21 Non-Homogeneous Equation with Constant Coefficients}
\begin{align*}
    \star a_n y^n+a_{n-1}y{n-1}+\text{...}+a_1y'+a_0 y&=Q(x)\\
    &\neq 0 \leftarrow \text{Non-Homogeneous}\\
    y(x)&=y_c(x)+y_p(x)\\
    y_c(x)&=\text{Solution to Homogeneous}\\
    y_p(x)&=\text{Particular solution to solve } Q(x)\\
\end{align*}
\begin{imp:defn}{Method of Undetermined Coefficients (Muc)}{} Only used when $Q(x)$ consists of sums or combinations of:$$a, x^k,e^{ax},\sin (\beta x). \cos (\beta x)$$
\end{imp:defn}
\begin{case}
No term in $Q(x)$ is similar to a term in $y_c(x)$ to the differential equation.\\
$y_p$ will be linear combination of terms presents in $Q(x)$ and all derivatives of those terms.
\begin{align*}
    y''+3y'+2y&=12e^x
    \end{align*}
\begin{step} Always find your complementary equation and build $y_c$ first
    \begin{align*}
    m^2+3m+12&0\\
    (m+2)(m+1)&=0 \rightarrow m=-1, m=-2\\
    y_c&=c_qe^{-x}+c_2e^{-2x}
    \end{align*}
\end{step}
\begin{step} Check for matching (Up to a Coefficient terms between $y_c$ and $Q(x)$
    \begin{align*}
    \text{No Matching Terms}
    \end{align*}
\end{step}
\begin{step} Find the shape of $y_p$
    \begin{align*}
    Q(X)&=12e^x\\
    y_p&=Ae^x\\
    y_p'=Ae^x\\
    y_p''=Ae^x
    \end{align*}
\end{step}
\begin{step} Take (enough) derivatives of $y_p$ so we can substitute back into the original Differential Equation
    \begin{align*}
    y_p''+3y_p'+2y_p&=12e^x\\
    Ae^x+3Ae^x+2Ae^x&=12e^x
    \end{align*}
\end{step}
\begin{step} Solve for Constants and Replace in $y_p$
    \begin{align*}
    6Ae^x&= 12e^x \rightarrow a=2, y_p=2e^x
    \end{align*}
\end{step}
\begin{step} Add $y_c$ and $y_p$ to form general solution for $y$
    \begin{align*}
    y&= c_1e^{-x}+c_2e^{-2x}+2e^x\\
    y_c&=c_1e^{-x}+c_2e^{-2x}\\
    y_p&=2e^x
\end{align*}
\end{step}
\end{case}
\begin{case}
Let $u(x)$ be a term in $y_c$. If $Q(x)$ contains a term of form $x^k+u(x), k\get 0$ then $y_p$ in linear combination of $x^{k+1} * u(x)$ and all linearly independent derivatives.
\begin{align*}
    y''+y&=\sin (x)\\
    m^2+1&=0 \rightarrow m=\pm i, \alpha =0, \beta =1\\
    y_c&= c_1 \cos (x) + c_2 \sin(x)\\
    u(x)&= \sin x \text{ then } x^\textcolor{red}{0}*\sin (x) \text{is in Q(x)}\\
    \textcolor{red}{k}&= \textcolor{red}{0}
    \end{align*}
\begin{step} $y_p$ has $x^1*\sin (x)$ and all Linear Independent Derivatives.
    \begin{align*}
    \frac{d}{dx} [q]&= x*\cos (x) +  \sin (x)\\
    q''&= -x*\sin (x)+ \cos x + \cos x\\
    q'''&= -x\cos x + - \sin x +-2 \sin x\\
    y_p&= Ax * \sin x +B x \cos x +C \sin x + D \cos x\\
    \textcolor{red}{\text{Recall }y}&=\textcolor{red}{y_c+y_p}\\
    y_c &\leftarrow\text{ has $\sin x$ and $\cos x$, so we don't need them in $y_p$ (lazy but precise)}\\
    y_p&=Ax\sin x +Bx\cos x
    \end{align*}
\end{step}
\begin{step} Take the derivatives of $y_p$, put back into original differential equation solve for $A,B$
    \begin{align*}
    y_p'&= Ax\cos x+A\sin x+-Bx\sin x+B\cos x\\
    &= (Ax+B)\cos x+(A-Bx) \sin x\\
    \text{and } y_p''&= (2A-Bx)\cos x - (Ax+2B)\sin x\\
    &= [(2A-Bx)\cos x - (Ax-2B)\sin x] + [Ax\sin x +Bx\cos c]\\
    &= \sin x\\
    2A\cos x + -2B\sin x &= \sin x + 0\cos x\\
    2A\cos x &= 0\cos x\\
    2A&= 0\\
    -2B\sin x&= 1\sin x\\
    -2B&=1\\
    A&=0\\
    B&= -\frac{1}{2}\\
    y_p&= -\frac{1}{2}x\cos x\\
    y&= c_1\cos x+c_2 \sin x -\frac{1}{2}x \cos x
\end{align*}
    
\end{step}
\end{case}
\begin{case}
Both Case 1, Case 2 specifically multiple roots and matching pieces in $Q(x)$
\begin{align*}
    y''-2y'+y&=e^x\\
    m^2-2m+1&=0 \rightarrow m=1, \text{multiple }2\\
    y_c&= \left( c_1+c_2x\right) e^x
\end{align*}
\begin{step} Since $c_1+c_2x$ is not constant, we need $y_p$ to have the form $x^2e^x$ and all linear independent terms
\begin{align*}
    q&=x^2e^x\\
    q'&=2xe^x+x^2e^x\\
    q'&=2xe^x \leftarrow \text{Don't need $x^2e^x$ because we have it in $q(x)$}\\
    q'&= \leftarrow \text{Don't need $2xe^x$ because we have it in $y_c$}\\
    y_p&=Ax^2e^x\\
    y_p'&=\left( Ax^2+2Ax\right) e^x\\
    y_p''&= \left( Ax^2+4Ax+2A \right) e^x\\
    y_p''-2y_p'+y_p&=e^x\\
    (Ax^2+4Ax+2A)e^x\\
    -2(Ax^2+2Ax)e^x\\
    (Ax^2)e^x\\
    (0Ax^2+0Ax+2A)e^x&= e^x(0x^2+0x+1)\\
    2A&=1\\
    A&=\frac{1}{2}\\
    y&=(c_1+c_2x)e^x+\frac{1}{2}x^2e^x\\
    y&=(c_1+c_2x+\frac{1}{2}x^2)e^x
\end{align*}
\end{step}
\end{case}