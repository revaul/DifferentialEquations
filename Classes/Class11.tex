% !TEX root = /Users/us2009801/Documents/GitHub/DifferentialEquations/main.tex
\chapter{Class 11 - Exam Review - Thursday, March 2\ts{nd}, 2017}
\begin{rvw}
Exam 2 Review\\
The Exam will cover:
\begin{itemize}
    \item Integrating Factors
    \begin{itemize}
        \item 1\ts{st} Order Linear Differential Equations
        \begin{itemize}
            \item $\frac{dy}{dx} +P(x)*y=Q(x)$
            \item $IF = e^{\int P(x)\,dx}$
        \end{itemize}
        \item Bernulli Equation
        \begin{itemize}
            \item $\frac{dy}{dx}+P(x)*y=Q(x)*y^n$
            \item $BIF = (1-n)n^{-n} \rightarrow $ Converts into 1\ts{st} Order Differential Equation
        \end{itemize}
    \end{itemize}
    \item \S 15 Applications
    \begin{itemize}
        \item Tanks, Temperature (Homework)
    \end{itemize}
    \item \S 18 Complex Numbers
    \item \S 19,20 Linear Independence
    \begin{itemize}
        \item n\ts{th} order constant Coefficient Differential Equations
        \item Characteristic Equations - 3 Cases
    \end{itemize}
    \item \S 21 Nonhomogeneous Constant Coefficient Differential Equations - 3 Cases $y=y_c+y_p$
\end{itemize}
\begin{ex}
\S 21.5\\
\begin{align*}
    y''+3y'+2y&=e^{ix}\\
    m^2+3m+2&=0\\
    m&=-2\\
    m&=-1\\
    y_c&= c_1 e^{-2x}+c_2e^{-x}
\end{align*}
\begin{note}
Guess For the Future:
\end{note}
\begin{align*}
    q&=e^{ix}\\
    q'&=ie^{ix}\\
    q''&=i^2e^{ix}=-e^{ix}\\
    i&=\sqrt{-1}\\
    i^2&=-1\\
    y_p&=Ae^{ix}+Bie^{ix}\\
    y_p'&=Aie^{ix}+-Be^{ix}\\
    y_p''&=-Ae^{ix}-Bie^{ix}
\end{align*}
\begin{note}
Add these
\end{note}
\begin{align*}
    e^{ie}&=e^{ix}(2A-3B-A)+ie^{ix}(2B+3A-B)\\
    1&=A-3B\\
    1&=10A\\
    A&=\frac{1}{10}\\
    0&=B+3A\\
    B&=-3A\\
    B&=\frac{-3}{10}\\
    y_p&=\frac{1}{10}e^{ix}+\frac{-3}{10}ie^{ix}\\
    y&=c_1e^{-2x}+c_2e^{-x}+\frac{1}{10}e^{ix}-\frac{3}{10}ie^{ix}\\
\end{align*}
\end{ex}
\begin{ex}
\S 21.19
\begin{align*}
    y''+3y'+2y&=e^{-2x}+x^2\\
    y_c&=c_1e^{2x}+c_2e^{-x}
\end{align*}
\begin{note}
Use the idea of superposition, namely $Q(x)=Q_1(x)+Q_2(x)$ and solve for $Q_1, Q_2$ Separately, then add together. Consider first:
\end{note}
\begin{align*}
    Q_1(x)&=e^{-2x}\\
    q&=xe^{-2x}\\
    q'&=-2xe^{-2x}+e^{-2x}\\
    y_{p_1}&= Axe^{-2x}\\
\end{align*}
\begin{note}
We use this for $Q_1(x)$. Then do some for $Q_2(x)=x^2$ $$y_{p_2}=Ax^2+Bx+C$$ Use this for $Q_2(x)$. Then $y_p=y_{p_1}+y_{p_2}$
\end{note}
\end{ex}
\begin{ex}
\S 21.13
\begin{align*}
    y''+y'&=x^2+2x\\
    y_1&=c_1+c_2e^{-x}\\
    y_p&=Ax^2+Bx
\end{align*}
\begin{note}
Don't need C, Because $c_1$ in $y_c$
\end{note}
\end{ex}
\begin{ex}
\S 15.5 A tank initially contains 100 gal of brine whose salt concentration is? lb per gal. Brine whose salt concentration is 2 lb per gal flows into the tank at the rate of 3 gal per min. The mixture flows out at the rate of 2 gal per min. Find the salt content of the brine and its concentration at the end of 30 min. Hint. After 30 min, the tank contains 130 gal of brine.
\end{ex}
\end{rvw}